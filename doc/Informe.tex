\documentclass[a4paper,10pt]{article}
\usepackage[spanish]{babel} % Paquetes de idioma
\usepackage[latin1]{inputenc} % Paquetes de idioma
\usepackage{graphicx} % Paquete para ingresar gráficos
\usepackage{grffile}
\usepackage{hyperref}
\usepackage{fancybox}
\usepackage{amsmath}
\usepackage{listings}

% Encabezado y Pié de página
\input{EncabezadoyPie.tex}
% Carátula del Trabajo
\title{ \author{} % Lo pongo para que el warning no moleste :p
\setlength{\unitlength}{1cm} %  Especifica la unidad de trabajo
\thispagestyle{empty}

\begin{picture}(18,0)
\put(0,0){\includegraphics[width=1.5cm, height=3cm]{Logo1.png}}

\put(10.5,0){\includegraphics[width=3cm, height=3cm]{Logo2.png}}

\end{picture}
\\[1.5cm]
\begin{center}
	\textbf{{\Huge Facultad de Ingenier\'ia \\ Universidad de Buenos Aires}}\\[2cm]
	{ 75.59 T\'ecnicas de Programaci\'on Concurrente I}\\[0.5cm]
	{ Trabajo Pr\'actico N${^\circ}$2}\\[2.5cm]
\end{center}

\begin{flushleft}
	\textbf{Integrantes:} \\[1cm]

	\begin{tabular}{|c|c|c|}
		\hline
		\textbf{\normalsize Padr\'on} & \textbf{\normalsize Nombre} & \textbf{\normalsize Email} \\
		\hline
		\normalsize 87991 & \normalsize Ramonda, Juan Pablo & \normalsize juanpr@gmail.com \\
		\hline
		\normalsize 89542 & \normalsize Hurtado, Pablo Nicol\'as & \normalsize hurtado.pani@gmail.com \\
		\hline
		\normalsize 89579 & \normalsize Torres Feyuk, Nicol\'as R. Ezequiel & \normalsize ezequiel.torresfeyuk@gmail.com \\
		\hline
	\end{tabular}
\end{flushleft}
\date{} % Hace que no se imprima la fecha en la cual se compilo el .tex
 }

\begin{document}
	\maketitle % Hace que el título anterior sea el principal del documento
	\newpage

	\tableofcontents % Esta línea genera un indice a partir de las secciones y subsecciones creadas en el documento
	\newpage

	\section{Introducci\'on}
		El presente trabajo consiste en el desarrollo de una aplicaci\'on {\it concurrente}. El objetivo de la misma
		es permitir el almacenamiento de datos centralizado y su posterior recuperaci\'on por parte de los clientes. \\
		\indent Debido a que la materia apunta a obtener conocimientos sobre los mecanismos de concurrencia vistos hasta el momento en la c\'atedra, 
    el proyecto debe correr en una \'unica computadora y funcionar bajo un ambiente \emph{Unix/Linux} dado que los mecanismos vistos son los 
    implementados por \emph{System V}. \\
		\indent La implementaci\'on de la aplicaci\'on consiste en un esquema \emph{Cliente-Servidor}. El servidor se encuentra escuchando peticiones
		del cliente. Los clientes deben conectarse al servidor, y luego de esto pueden utilizar los servicios provistos por el mismo.
		\vspace{0.5cm}

	\section{Modo de Operaci\'on}

		\subsection {C\'omo compilar y correr la aplicaci\'on}
			Para poder correr la aplicaci\'on correctamente, lo primero que debe realizarse es compilar el programa. Para esto se ha provisto de un
			\emph{Makefile} que se encagar\'a de realizar el proceso de compilaci\'on. \\
			\indent Para compilar la aplicaci\'on servidor se debe ingresar el siguiente comando:
			\begin{verbatim}
				$ make server
			\end{verbatim}
			\indent Para compilar la aplicaci\'on cliente se debe ingresar el siguiente comando:
			\begin{verbatim}
				$ make client
			\end{verbatim}
			\indent En el caso de que desee compilar todas las aplicaciones juntas puede hacerlo por medio de alguno de los siguientes comandos:
			\begin{verbatim}
				$ make server client 
				$ make all
			\end{verbatim}

			\subsection{Casos de Uso}

			Como bien se dijo, la aplicaci\'on est\'a compuesta por un \emph{Servidor} que escucha peticiones y \emph{Clientes} que realizan las mismas. 
			De esta forma, el \emph{Servidor} es un proceso que no tiene interacci\'on con el usuario. En cambio, el \emph{Cliente} si tiene interacci\'on
			con el usuario. La aplicaci\'on \emph{Cliente} despliega un men\'u con las tareas que puede realizar el mismo. A continuaci\'on se 
			explica el modo de uso de cada una de las opciones del men\'u: 

			\begin{itemize}
                                
                \item \textbf{1-Leer un Registro:} Mediante esta opci\'on, el cliente le indica al servidor qu/'e registro desea recuperar de la base de datos.

                Precondici\'on: El servidor se encuentra corriendo. 
                \begin{itemize}
                    \item El usuario solicita la lectura de un registro. 
                    \item El programa cliente pregunta el n\'umero de registro deseado. 
		    \item El usuario ingresa el n\'umero de registro. 
                    \item El programa cliente lo env\'ia al servidor.
                    \item El servidor recibe el pedido, valida el n\'umero pedido y responde enviando el registro solicitado o error (E1). 
		    \item El cliente recibe la respuesta y muestra el registro al usuario.
                \end{itemize}
                
                E1) El n\'umero de registro es inv\'alido. Corresponde a un registro inexistente en la base de datos. 
                    \begin{itemize}
                        \item El programa cliente muestra un mensaje de error y vuelve al men\'u de selecci\'on de operaci\'on.
				    \end{itemize}
                
                \item \textbf{2-Agregar un registro a la Base de Datos:} Mediante esta opci\'on, el cliente le indica al servidor que desea agregar un registro. 
                Precondici\'on: El servidor se encuentra corriendo.
                \begin{itemize}
				\item El usuario solicita agregar un registro a la base de datos. 
				\item El programa cliente pregunta el campo nombre del nuevo registro. 
				\item El usuario ingresa el nombre.
				\item El programa cliente valida el tama~no del nombre (E1).
				\item El programa cliente pregunta el campo direcci\'on del nuevo registro. 
				\item El usuario ingresa la direcci\'on.
				\item El cliente valida el tama~no de la direcci\'on (E1).
				\item El programa cliente pregunta el campo itel\'efono del nuevo registro. 
				\item El usuario ingresa el tel\'efono. 
			\item El cliente valida el tama~no del tel\'efono. (E1).
			\item El cliente env\'ia el registro a agregar al servidor.

			\item El servidor recibe el mensaje, agrega el registro a la base de datos, y env\'ia al cliente la respuesta. 
		    \item El cliente recibe el mensaje y notifica al usuario del \'exito de la operaci\'on.
			\item El cliente muestra el men\'u principal.
                \end{itemize}

                E1) El tama\~no del campo es inv\'alido. 
                \begin{itemize}
                    \item El programa cliente muestra un mensaje de error y solicita nuevamente el ingreso del valor del campo. 
			\item El usuario debe ingresar el valor del campo hasta que la validaci\'on sea correcta.
                \end{itemize}
                
                \item \textbf{3-Modificar un registro:} Mediante esta opci\'on, se permite la modificaci\'on de un registro de la base de datos. 
				
                \begin{itemize}
			\item El usuario solicita la modificaci\'on de un registro. 
			\item El cliente pregunta el n\'umero de registro a modificar.
			\item El usuario ingresa el n\'umero.
			\item El cliente pregunta el valor del campo nombre del registro. 
			\item El usuario ingresa el nombre.
			\item El programa cliente valida el tama\~no del nombre (E1).
			\item El cliente pregunta el valor del campo direcci\'on del registro 
			\item El usuario ingresa la direcci\'on.
			\item El programa cliente valida eltama~no del campo direcci\'on (E1).
			\item El cliente pregunta el valor del campo tel\'efono del registro 
			\item El usuario ingresa el tel\'efono. 
			\item El programa cliente valida el tama~no del campo tel\'efono (E1).
			\item El cliente notifica al usuario el resultado de la operaci\'on.
			\item El cliente muestra el men\'u principal.

			E1) El tama~no del campo es inv\'alido. 
			\begin{itemize}
                    \item El programa cliente muestra un mensaje de error y solicita nuevamente el ingreso del valor del campo. 
			\item El usuario debe ingresar el valor del campo hasta que la validaci\'on sea correcta.
			\end{itemize}
		\end{itemize}
                \item \textbf{4-Eliminar un registro de la Base de Datos:} Mediante esta opci\'on, el cliente puede eliminar un registro de la base de datos. 
                   
                Precondici\'on: El servidor se encuentra corriendo. 
				
                \begin{itemize}
                   \item El usuario solicita al programa cliente la eliminaci\'on de un registro. 
                   \item El programa cliente solicita el n\'umero de registro que se desea eliminar. 
                  \item El programa cliente envía la petici\'on de eliminaci\'on al servidor. 
                   \item El servidor realiza la petici\'on y env\'ia el resultado al cliente. 
			\item El cliente recibe la respuesta del servidor y la muestra al usuario (E1).
			\item El cliente regresa al men\'u principal.
                \end{itemize}
               E1) El n\'umero de registro ingresado no existe. 
		\begin{itemize}
			\item El cliente notifica el error al usuario.
			\item El cliente regresa al men\'u principal.
		\end{itemize}
		\item \textbf{5-Salir de la Aplicaci\'on:} Mediante esta opci\'on, el cliente abandona la aplicaci\'on.

                Precondici\'on: El servidor se encuentra corriendo. 
                
                \begin{itemize}
                    \item El usuario solicita salir del programa cliente.
                    \item Finaliza el programa cliente.
                \end{itemize}
			\end{itemize}

	\section{An\'alisis de la Soluci\'on}
		\subsection{Divisi\'on de la Aplicaci\'on en Procesos}
			Debido a la naturaleza de la soluci\'on adoptada, se pueden distinguir f\'acilmente los siguientes procesos:

			\begin{itemize}
				\item \textbf{Servidor:} Este proceso se encarga de escuchar las peticiones los clientes. Mientras ning\'un usuario le env\'ie ning\'un 
				mensaje, el mismo se bloquea. Cuando un cliente le env\'ia alg\'un mensaje, despierta al mismo y este se encarga de responderle al 
				usuario.
				\item \textbf{Cliente:} Al abrir un usuario una aplicaci\'on Cliente, se crea un nuevo proceso el cual puede comunicarse con el servidor.
				El Cliente env\'ia peticiones al servidor esperando una respuesta del mismo.
			\end{itemize}
\subsection{Comunicaci\'on entre Procesos} A continuaci\'on se detalla como es la comunicaci\'on entre los procesos seg\'un los casos de uso: \begin{itemize}
				\item \textbf{Leer Registro de la Base de Datos:}	
					\begin{itemize} \item El servidor y un cliente se comunican.
						\item El cliente env\'ia un mensaje al servidor para leer un registro. 
						\item El servidor recibe el mensaje y contesta al cliente con el registro solicitado o un error.
					\end{itemize}
				\item \textbf{Agregar Registro al Servidor:} 
					\begin{itemize}
						\item El servidor y un cliente se comunican.
						\item El cliente env\'ia un mensaje al servidor para agregar un registro a la base de datos, junto con el registro a agregar. 
						\item El servidor recibe el mensaje y contesta con el resultado de la operaci\'on. 
					\end{itemize}
				\item \textbf{Modificar un Registro:}
					\begin{itemize}
						\item El servidor y un cliente se comunican.
						\item El cliente env\'ia un mensaje al servidor pidiendomodificar un registro, con el registro modificado. 
						\item El servidor recibe el mensaje y contesta al cliente con el resultado de la operaci\'on. 
						 \end{itemize} 
\item \textbf{Eliminar un registro de la base de datos:} \begin{itemize}
						\item El servidor y un cliente se comunican.
						\item El cliente env\'ia un mensaje al servidor pidi\'endo la eliminaci\'on de un registro de la base de datos junto con el n\'umero de registro a eliminar.
						\item El servidor le responde con el resultado de la operaci\'on. 
					\end{itemize}
			\end{itemize}
	\section{Protocolo de Comunicaci\'on}
	A continuaci\'on se detalla el protocolo utiizado en la aplicaci\'on.
	Para transmitir mensajes se utiliz\'o una estructura con los siguientes elementos:
	\begin{itemize}
		\item mtype: c\'odigo necesario por el mecanismo de comunicaci\'on para identificar el receptor del mensaje.	
		\item pid: pid del cliente que env\'ia el mensaje al servidor.
		\item numReg: n\'umero de registro de la base de datos por el cual se solicita una operaci\'on (lectura/modificaci\'on/eliminaci\'on).
		\item reg: registro de la base de datos (en caso de querer agregar/modificar).
	\end{itemize}	
	Las acciones posibles y los valores para las mismas son: 
	\begin{enumerate}
		\item{\bf Conexi\'on}: Este comando se env\'ia autom\'aticamente cuando se abre un cliente.
			\begin{itemize}
				\item Cliente:
				\begin{itemize}
					\item[]mtype  = 1
					\item[]pid  = pidCliente
					\item[]comando  = CON
					\item[]numReg  = indefinido
					\item[]reg  = indefinido
				\end{itemize}
				\item Servidor:
				\begin{itemize}
					\item[]mtype = pidCliente 
					\item[]pid = pidCLiente
					\item[]comando = CONOK
					\item[]numReg = indefinido
					\item[]reg = indefinido
				\end{itemize}
			\end{itemize}
		\item{\bf Desconexi\'on}: Este comando se env\'ia autom\'aticamente cuando se desea cerrar un cliente.
			\begin{itemize}
				\item Cliente:
				\begin{itemize}
					\item[]mtype = 1
					\item[]pid = pidCliente
					\item[]comando = DESCON 
					\item[]numReg = indefinido
					\item[]reg = indefinido
				\end{itemize}
				\item Servidor:
				\begin{itemize}
					\item[]mtype = pidCliente 
					\item[]pid = pidCLiente
					\item[]comando = DESCONOK 
					\item[]numReg = indefinido
					\item[]reg = indefinido
				\end{itemize}
			\end{itemize}
		\item{\bf Lectura de un registro de la BD}: Comando ejecutado cuando se desea consultar un registro de la base de datos.
			\begin{itemize}
				\item Cliente:
				\begin{itemize}
					\item[]mtype = 1
					\item[]pid = pidCliente
					\item[]comando = LEERRG 
					\item[]numReg = n\'umero de registro a leer
					\item[]reg = indefinido
				\end{itemize}
				\item Servidor:
				\begin{itemize}
					\item[]mtype = pidCliente 
					\item[]pid = pidCLiente
					\item[]comando = LEERRGOK
					\item[]numReg = indefinido
					\item[]reg = registro leido
				\end{itemize}
			\end{itemize}

		\item{\bf Modificaci\'on de un registro de la BD}: Comando ejecutado cuando se desea modificar un registro de la base de datos.
			\begin{itemize}
				\item Cliente:
				\begin{itemize}
					\item[]mtype = 1
					\item[]pid = pidCliente
					\item[]comando = MODRG 
					\item[]numReg = n\'umero de registro a modificar
					\item[]reg = registro a reemplazar
				\end{itemize}
				\item Servidor:
				\begin{itemize}
					\item[]mtype = pidCliente 
					\item[]pid = pidCLiente
					\item[]comando = MODREGOK
					\item[]numReg = indefinido
					\item[]reg = indefinido
				\end{itemize}
			\end{itemize}

		\item{\bf Baja de un registro de la BD}: Comando ejecutado cuando se desea eliminar un registro de la base de datos.
			\begin{itemize}
				\item Cliente:
				\begin{itemize}
					\item[]mtype = 1
					\item[]pid = pidCliente
					\item[]comando = BAJARG 
					\item[]numReg = n\'umero de registro a eliminar
					\item[]reg = indefinido
				\end{itemize}
				\item Servidor:
				\begin{itemize}
					\item[]mtype = pidCliente 
					\item[]pid = pidCLiente
					\item[]comando = BAJARGOK
					\item[]numReg = indefinido
					\item[]reg = indefinido
				\end{itemize}
			\end{itemize}
		\end{enumerate}
	Al no concluir exitosamente una operaci\'on, el servidor responde con un mensaje de error, el cual se detalla a continuaci\'on: \\
		Servidor:
			\begin{itemize}
				\item[]mtype = pidCliente 
				\item[]pid = pidCLiente
				\item[]comando = ERROR
				\item[]numReg = indefinido
				\item[]reg = indefinido
			\end{itemize}


%	\subsection{Problemas Conocidos de Concurrencia en el Trabajo}
%	Respecto a las transferencias, se podr\'ia comparar con el 
%	problema de \emph{Consumidor Productor}, con la excepci\'on de que habr\'ia un solo consumidor y un solo productor. \\
%	Sin embargo, se tienen los mismos problemas que en el modelo, ya que el lector debe esperar a que el escritor
%	 ``produzca'' 
%	y el escritor debe esperar a que el lector ``consuma''.
%	
%	Respecto a la comunicaci\'on entre cliente y servidor, la comunicaci\'on tambi\'en se parece al problema de \emph
%	{Cosumidor Productor},
%	 con la
%	 excepci\'on de que puede haber muchos productores y un solo consumidor. En el caso de las respuestas del servidor
%	 a cada
%	 cliente, solo hay un productor y un consumidor.\\
%	En este caso, el lector siempre debe esperar a que el escritor ``produzca''.
%	
%	Respecto a la lectura del archivo para la transferencia, se podr\'ia comparar con el modelo de \emph{Lectores
%	 Escritores}, aunque en la aplicaci\'on se pueden tener varios lectores para el archivo a transferir
%	y un solo escritor para el archivo nuevo. \\Aunque, an\'alogamente 
%	al caso anterior, se pueden presentar los problemas caracter\'isticos de este modelo, es decir, podr\'ia darse el caso
%	de que se desee modificar de forma externa al archivo de lectura mientras est\'a siendo leido. Para el archivo de 
%	escritura, 
%	podr\'ia ser que se quiera leer o escribir el archivo mientras se est\'a escribiendo.
%	
%	\subsection{Mecanismos de Concurrencia Utilizados}
%		A continuaci\'on se exhiben los mecanismos de concurrencia utilizados en la comunicaci\'on entre los procesos:
%		\subsubsection{Env\'io de Mensajes entre Servidor y Cliente}
%			Los clientes se comunican con el servidor enviando mensajes con la tarea que desean realizar. Dado que el servidor 
%			debe responder al cliente sobre el estado de la operaci\'on realizada, los mecanismos posibles para establecer esta 
%			comunicaci\'on podr\'ian ser \emph{Fifos} o \emph{Memoria Compartida}. No se podr\'ia haber utilizado \emph{pipes} dado 
%			que el cliente deber\'ia ser hijo del servidor, cosa que no ocurre. \\
%			\indent Se decidi\'o utilizar \emph{Fifos} para establecer esta comunicaci\'on dado que los mensajes que env\'ia el cliente
%			tienen distintos tama\~nos seg\'un la tarea a realizar, y en algunos casos como a la hora de compartir archivos, var\'ian 
%			seg\'un la longitud del path del archivo compartido. \\ \\
%			\indent De esta forma, el servidor posee una \emph{Fifo} para recibir mensajes del cliente y cada cliente posee una \emph{Fifo}
%			para escribir mensajes al servidor. Con esta implementaci\'on surge un inconveniente: Una \emph{Fifo} es un mecanismo de 
%			concurrencia unidireccional que asegura el sincronismo entre procesos. Sin embargo, dado que todos los clientes conectados poseen
%			la \emph{Fifo} de lectura del servidor abierta, ya no existe sincronismo entre los procesos.  \\
%			\indent Para solucionar este problema, se agrega un sem\'aforo. Este sem\'aforo se encarga de bloquear al servidor mientras que no 
%			haya alg\'un cliente que intente escribir alg\'un mensaje. El cliente que desea escribir un mensaje incrementa el sem\'aforo y 
%			despierta al servidor para procesar el mensaje. \\ \\
%			\indent Los mensajes que el servidor env\'ia a cada cliente son entregados tambi\'en por medio de \emph{Fifos}. En este caso, el 
%			servidor posee una \emph{Fifo} por cada cliente que se encuentre conectado, por lo cual no existen problemas de sincronismo dado que
%			este mecanismo s\'olo es compartido entre el servidor y un cliente particular.
%	
%			\subsubsection{Transferencia de Archivos}
%			Para realizar la transferencia de archivos se pod\'ia elegir nuevamente entre \emph{Fifos} y \emph{Memoria Compartida}. En este caso,
%			se eligi\'o el mecanismo de concurrencia de \emph{Memoria Compartida}. El motivo de esta elecci\'on es que de esta forma el archivo
%			compartido se escribe en memoria y no en un archivo en disco como se lo hace utilizando \emph{Fifos}. De esta forma, se mejora el 
%			rendimiento de la aplicaci\'on. \\
%			\indent Dado que el tama\~no del archivo puede llegar a ser muy grande, el archivo se escribe de forma particionada en la 
%			\emph{Memoria Compartida}. \\
%			\indent Para asegurar la sincronizaci\'on entre escrituras y lecturas, se emplean dos sem\'aforos. Un sem\'aforo bloquea 
%			la lectura de la \emph{Memoria}. Esto se requiere en caso de que otro proceso se encuentre escribiendo la misma. Al terminar
%			de escribir, el proceso desbloquea la \emph{Memoria Compartida} y el proceso que deseaba realizar una lectura se libera para
%			poder completarla. El otro sem\'aforo bloquea la escritura de la \emph{Memoria}. Esto se requiere en caso de que un proceso
%			se encuentre leyendo la memoria a la hora de que otro quiera escribirla. En este caso, el proceso que escribe se suspende 
%			hasta que el proceso que lee incremente el sem\'aforo, lo cual se da cuando termina de leer.
%			
		% Esto no iria porque no usamos Locks sobre el archivo de Log
		%	\subsubsection{Escritura sobre el Logger} 
		% 		Al ejecutarse la aplicaci\'on en \emph{Modo Debug}, muchos procesos podr\'ian escribir el archivo al mismo tiempo. Esto
		%	puede provocar que los mensajes se mezclen entre s\'i y el archivo de log quede ilegible.
		%	Para solucionar este inconveniente, se agrega un lock de escritura sobre el archivo de log cada vez que un proceso logea
		%	alg\'un mensaje. El mismo se retira al terminarse la escritura sobre el mismo, dejando que cualquier otro proceso pueda 
		%	escribir otro mensaje.

%	\section{Diagramas}
%		
%		\subsection{Diagramas de Estado}
%			\begin{figure}[!htpb]
%				\centering
%				\includegraphics[width=11cm]{DEstadoVidaCliente.jpeg}
%				\caption{Diagrama de Estado del Proceso Cliente} \label{Img001}
%			\end{figure}
%	
%			\begin{figure}[!htpb]
%				\centering
%				\includegraphics[width=11cm]{DEstadoVidaServidor.jpeg}
%				\caption{Diagrama de Estado del Proceso Servidor} \label{Img002}
%			\end{figure}
%	
%			\begin{figure}[!htpb]
%				\centering
%				\includegraphics[width=11cm]{DEstadoVidaTransferidor.jpeg}
%				\caption{Diagrama de Estado del Proceso Transferidor}
%			\end{figure}
%	
%		\newpage
%		\subsection{Diagrama de Clases}
%			\begin{figure}[!htpb]
%				\centering
%				\includegraphics[width=12cm]{dclases.png}
%				\caption{Diagrama de Clases de la Aplicaci\'on ConcuShare}
%			\end{figure}
		
			
			
		
		

				
\end{document}
